\documentstyle[12pt]{article}

\begin{document}

\parindent=1em
\parskip=1.5ex

\begin{center}

\noindent{\large\bf Papers about Text Generation and Explanation}\\[1.5ex]
\noindent{\large\bf from the EXPECT and EES Projects}\\[1.5ex]
\noindent{\large\bf (1982-1993)}\\[1.5ex]

{\bf
\hspace{1mm}\\
Information Sciences Institute\\
University of Southern California \\
4676 Admiralty Way \\
Marina del Rey, CA 90292 \\
(310) 822-1511 \\
expect@isi.edu \\
}

\end{center}



\begin{thebibliography}{10}



\bibitem{SwartoutAIJ}
William~R. Swartout.
\newblock {XPLAIN: A System for Creating and Explaining Expert Consulting
  Systems}.
\newblock {\em Artificial Intelligence}, 21(3):285--325, September 1983.
\newblock Also available as ISI/RS-83-4.

\noindent\hspace*{\itemindent}{\leftskip=0.1in\rightskip=0.1in\hrulefill}

\bibitem{ClanceyShortliffe-book}
William~R. Swartout.
\newblock {Explaining and Justifying Expert Consulting Programs}.
\newblock In {\em Readings in Medical Artificial Intelligence: The First
  Decade}. Addision-Wesley, 1984.

\noindent\hspace*{\itemindent}{\leftskip=0.1in\rightskip=0.1in\hrulefill}

\bibitem{Swartout-AFIPS85}
William~R. Swartout.
\newblock {Knowledge needed for Expert System Explanation}.
\newblock In {\em AFIPS Conference Proceedings}, volume~54, pages 93--98.
  National Computer Conference, 1985.

\noindent\hspace*{\itemindent}{\leftskip=0.1in\rightskip=0.1in\hrulefill}

\bibitem{EES-IJCAI85}
Robert Neches, William~R. Swartout, and Johanna~D. Moore.
\newblock {Explainable And Maintainable Expert Systems}.
\newblock In {\em Proceedings of the Ninth International Joint Conference on
  Artificial Intelligence}, volume One, pages 382--389. IJCAI-85, August 1985.

\noindent\hspace*{\itemindent}{\leftskip=0.1in\rightskip=0.1in\hrulefill}

\bibitem{EES}
Robert Neches, William~R. Swartout, and Johanna~D. Moore.
\newblock {Enhanced Maintenance and Explanation of Expert Systems Through
  Explicit Models of Their Development}.
\newblock {\em IEEE Transactions on Software Engineering},
  SE-11(11):1337--1351, November 1985.

{\leftskip=0.1in\rightskip=0.1in\begin{small}\par\noindent{\bf Abstract:\ 
  }Principled development techniques could greatly enhance the
understandability of expert systems for both users and system developers. 
Current systems have limited explanatory capabilities and present
maintenance problems because of a failure to explicitly represent the
knowledge and reasoning that went into their design.  This paper describes
a paradigm for constructing expert systems which attempts to identify that
tacit knowledge, provide means for capturing it in the knowledge bases of
expert systems, and apply it towards more perspicuous machine-generated
explanations and more consistent and maintainable system organization.
  \end{small}\par}
\noindent\hspace*{\itemindent}{\leftskip=0.1in\rightskip=0.1in\hrulefill}


\bibitem{Mann-Bates-et-al82}
William~C. Mann, Madeline Bates, Barbara~J. Grosz, David~D. McDonald,
  Kathleen~R. McKeown, and William~R. Swartout.
\newblock {Text Generation}.
\newblock {\em American Journal of Computational Linguistics}, 8(2):62--69,
  April-June 1982.

\noindent\hspace*{\itemindent}{\leftskip=0.1in\rightskip=0.1in\hrulefill}

\bibitem{ExplanationWorkshop}
William~R. Swartout.
\newblock {Report on Workshop on Automated Explanation Production}.
\newblock {\em ACM SIGART}, 1(85), 1983.

\noindent\hspace*{\itemindent}{\leftskip=0.1in\rightskip=0.1in\hrulefill}

\bibitem{Swartout-aamsi}
William~R. Swartout.
\newblock {Beyond XPLAIN: toward more explainable expert systems}.
\newblock In {\em Proceedings of the Congress of the American Association of
  Medical Systems and Informatics}, pages 102--106. American Association of
  Medical Systems and Informatics, 1986.

\noindent\hspace*{\itemindent}{\leftskip=0.1in\rightskip=0.1in\hrulefill}

\bibitem{McKeownSwartout87}
Kathleen~R. McKeown and William~R. Swartout.
\newblock {Language Generation and Explanation}.
\newblock In {\em Annual Reviews in Computer Science}, 1987.

\noindent\hspace*{\itemindent}{\leftskip=0.1in\rightskip=0.1in\hrulefill}

\bibitem{ParisMcKeown87}
C\'{e}cile~L. Paris and Kathleen~R. McKeown.
\newblock {Discourse Strategies for Describing Complex Physical Objects}.
\newblock In Gerard Kempen, editor, {\em Natural Language Generation: Recent
  Advances in Artificial Intelligence, Psychology, and Linguistics}. Kluwer
  Academic Publishers, Boston/Dordrecht, 1987.

\noindent\hspace*{\itemindent}{\leftskip=0.1in\rightskip=0.1in\hrulefill}

\bibitem{Paris-IJCAI87}
C\'{e}cile~L. Paris.
\newblock {Combining Discourse Strategies to Generate Descriptions to Users
  Along a Naive/Expert Spectrum}.
\newblock In {\em Proceedings of IJCAI-87}, pages 626--632, Milan, Italy, 1987.
  International Joint Conferences on Artificial Intelligence.
\newblock ISI Technical Report \# ISI-RR-93-311.

\noindent\hspace*{\itemindent}{\leftskip=0.1in\rightskip=0.1in\hrulefill}

\bibitem{ExplSurvey}
Johanna~D. Moore and William~R. Swartout.
\newblock {Explanation in Expert Systems: A Survey}.
\newblock Technical Report ISI/RR-88-228, USC/Information Sciences Institute,
  1988.

\noindent\hspace*{\itemindent}{\leftskip=0.1in\rightskip=0.1in\hrulefill}

\bibitem{Paris88-book}
C\'{e}cile~L. Paris.
\newblock {Explicit User Models in Text Generation: Tailoring Objects
  Descriptions to a Users' Level of Expertise}.
\newblock In Alfred Kobsa and Wolfgang Wahlster, editors, {\em User Models in
  Dialog Systems}, pages 200--232. Springer Verlag, Symbolic Computation
  Series, Berlin Heidelberg New York Tokyo, 1988.
\newblock ISI Technical Report \# ISI-RR-93-312.

\noindent\hspace*{\itemindent}{\leftskip=0.1in\rightskip=0.1in\hrulefill}

\bibitem{Paris-Wick-Swartout-Thompson-AAAI88-wkshp}
C{\'e}cile Paris, Michael Wick, William Thompson, and William Swartout.
\newblock {The Line of Reasoning vs The Line of Explanation}.
\newblock In C{\'e}cile Paris, Michael Wick, William Thompson, and William
  Swartout, editors, {\em {Proceedings of the AAAI'88 Workshop on
  Explanation}}, St Paul, Minnesota, August 1988. American Association for
  Artificial Intelligence.

\noindent\hspace*{\itemindent}{\leftskip=0.1in\rightskip=0.1in\hrulefill}

\bibitem{MooreParis88}
Johanna~D. Moore and C\'{e}cile~L. Paris.
\newblock {Constructing Coherent Texts Using Rhetorical Relations}.
\newblock In {\em Proceedings of the Tenth Annual Conference of the Cognitive
  Science Society}. Cognitive Science Society, August 1988.
\newblock Authors in alphabetical order.

\noindent\hspace*{\itemindent}{\leftskip=0.1in\rightskip=0.1in\hrulefill}

\bibitem{Paris-AAAI88-TP}
C\'{e}cile~L. Paris.
\newblock {Planning a text: can we and how should we modularize this process?}
\newblock In the {\it Proceedings of the AAAI-88 Workshop on Text Planning and
  Natural Language Generation\/}, August 1988.
\newblock Extended Abstract.

\noindent\hspace*{\itemindent}{\leftskip=0.1in\rightskip=0.1in\hrulefill}

\bibitem{Paris-CL88}
C\'{e}cile~L. Paris.
\newblock Tailoring {O}bject {D}escriptions to the {U}ser's {L}evel of
  {E}xpertise.
\newblock {\em Computational Linguistics}, 14 (3):64--78, September 1988.

\noindent\hspace*{\itemindent}{\leftskip=0.1in\rightskip=0.1in\hrulefill}

\bibitem{Swartout-Smoliar89}
William~R. Swartout and Stephen~W. Smoliar.
\newblock {Explanation: A Source of Guidance for Knowledge Representation}.
\newblock In K.~Morik, editor, {\em Knowledge Representation and Organization
  in Machine Learning}, volume 347 of {\em Lecture Notes in Artificial
  Intelligence}, pages 1--16. Springer-Verlag, Berlin, West Germany, 1989.

\noindent\hspace*{\itemindent}{\leftskip=0.1in\rightskip=0.1in\hrulefill}


\bibitem{Bateman-ParisIJCAI}
John~A. Bateman and C\'{e}cile~L. Paris.
\newblock {Phrasing a Text in Terms the User Can Understand}.
\newblock In {\em Proceedings of the Eleventh International Joint Conference on
  Artificial Intelligence}, pages 1511--1517, Detroit, Michigan, 1989.

{\leftskip=0.1in\rightskip=0.1in\begin{small}\par\noindent{\bf Abstract:\ }When
  humans use language, they show an essential, inbuilt responsiveness to their
  hearers/readers. When language is generated by machine, it is similarly
  necessary to ensure that that language is appropriate for its intended
  audience. Much of previous research on text generation and user modelling has
  focused on building a user model and selecting appropriate information from
  the knowledge base to present to the user. It is important, however, that the
  {\em phrasing\/} of a text be also tailored to the hearer -- otherwise it may
  be just as ineffective as texts which wrongly direct attention or which rely
  on knowledge that the hearer does not have. This research proposes a new
  mechanism which allows the text planning process to specifically tailor
  syntactic phrasing to the hearer type. This is done in the context of an
  expert system explanation facility that needs to produce explanations of the
  expert system's behavior for a variety of different users -- users who differ
  in goals, expectations, and expertise concerning both the expert system and
  its domain.\\ \\Authors in alphabetical order.\end{small}\par}
\noindent\hspace*{\itemindent}{\leftskip=0.1in\rightskip=0.1in\hrulefill}

\bibitem{Moore89-thesis}
Johanna~D. Moore.
\newblock {\em {A Reactive Approach to Explanation in Expert and Advice-Giving
  Systems}}.
\newblock PhD thesis, University of California, Los Angeles, 1989.

\noindent\hspace*{\itemindent}{\leftskip=0.1in\rightskip=0.1in\hrulefill}

\bibitem{MooreCHI89}
Johanna~D. Moore.
\newblock {Responding to ``Huh?'': Answering Vaguely Articulated Follow-Up
  Questions}.
\newblock In {\em Proceedings of the Conference on Human Factors in Computing
  Systems}, Austin, Texas, April 30 - May 4 1989.
\newblock ISI Technical Report \# ISI-RR-93-313.

\noindent\hspace*{\itemindent}{\leftskip=0.1in\rightskip=0.1in\hrulefill}

\bibitem{Moore-Swartout-IJCAI89}
J.~D. Moore and W.~D. Swartout.
\newblock {A Reactive Approach to Explanation}.
\newblock In {\em Proceedings of the Eleventh International Conference on
  Artificial Intelligence}, pages 1505--1510, Detroit, MI, August 1989.

\noindent\hspace*{\itemindent}{\leftskip=0.1in\rightskip=0.1in\hrulefill}

\bibitem{MooreParis89}
Johanna~D. Moore and C\'{e}cile~L. Paris.
\newblock {Planning text for advisory dialogues}.
\newblock In {\em Proceedings of the 27th Annual Meeting of the Association for
  Computational Linguistics}, pages 203--211. Association for Computational
  Linguistics, June 1989.
\newblock ISI Technical Report \# RS 89-236.

{\leftskip=0.1in\rightskip=0.1in\begin{small}\par\noindent{\bf Abstract:\
  }Explanation is an interactive process requiring a dialogue between
  advice-giver and advice-seeker. In this paper, we argue that in order to
  participate in a dialogue with its users, a generation system must be capable
  of reasoning about its own utterances and therefore must maintain a rich
  representation of the responses it produces. We present a text planner that
  constructs a detailed text plan, containing the intentional, attentional, and
  rhetorical structures of the text it generates.\\ \\ Authors in alphabetical
  order.\end{small}\par}
\noindent\hspace*{\itemindent}{\leftskip=0.1in\rightskip=0.1in\hrulefill}

\bibitem{Bateman-Paris-UM}
C\'{e}cile~L. Paris and John~A. Bateman.
\newblock {User Modeling and Register Theory: A congruence of concerns}.
\newblock ISI Technical Report \# ISI-RS-93-315, 1990.

{\leftskip=0.1in\rightskip=0.1in\begin{small}\par\noindent{\bf Abstract:\
  }Sophisticated computer systems using natural language to interact with
  people are now becoming widespread. These systems need to communicate with an
  increasingly varied user community, across an ever more extensive range of
  situations. Just as for human-human interaction, no single style of generated
  text is adequate across all user types and all situations. Generation systems
  can only be effective if they appropriately `tailor' their phrasing, text
  content, and organization according to the situation and to the abilities and
  requirements of the intended readers. This paper presents new work in
  `tailoring' that addresses the {\em phrasing problem\/}: how to best express
  the propositional content that has been chosen by a text planner, given a
  user and a situation. Importantly, this paper shows how relevant linguistic
  studies can be bought to bear on the problem of user modeling and tailoring.
  In particular, we would like to show that the concerns of register theory are
  very close to some of the concerns of user modeling, and that aspects of the
  theory can guide us in our studies in user modeling. Based on this specific
  linguistic theory, we propose a methodology to systematically study the
  problem of tailoring phrasing.\\ \\ Authors in alphabetical
  order.\end{small}\par}
\noindent\hspace*{\itemindent}{\leftskip=0.1in\rightskip=0.1in\hrulefill}

\bibitem{Paris-Catalina-Book}
C\'ecile~L. Paris, William~R. Swartout, and William~C. Mann, editors.
\newblock {\em {Natural Language Generation in Artificial Intelligence and
  Computational Linguistics}}.
\newblock Kluwer Academic Publishers, Boston/Dordrecht/London, 1990.

\noindent\hspace*{\itemindent}{\leftskip=0.1in\rightskip=0.1in\hrulefill}

\bibitem{Mittal-Paris-Manchester}
Vibhu~O. Mittal and C{\'e}cile~L. Paris.
\newblock {On the use of Analogies in Explanations in EES}.
\newblock In Nick Filer, editor, {\em {Proceedings of the 5th Workshop on
  Explanation}}, Manchester, UK, April 1990.

\noindent\hspace*{\itemindent}{\leftskip=0.1in\rightskip=0.1in\hrulefill}

\bibitem{Mittal-Paris-AAAI90-wkshp}
Vibhu~O. Mittal and C{\'e}cile~L. Paris.
\newblock {Analogical Explanations in the EES Framework}.
\newblock In Johanna~D. Moore and Michael~R. Wick, editors, {\em {Proceedings
  of the AAAI'90 Workshop on Explanation}}, pages 162 -- 172, Boston, MA,
  August 1990. American Association for Artificial Intelligence.

\noindent\hspace*{\itemindent}{\leftskip=0.1in\rightskip=0.1in\hrulefill}

\bibitem{MooreAAAI90}
Johanna~D. Moore and William~R. Swartout.
\newblock {Pointing: A way towards explanation dialogue}.
\newblock In {\em Proceedings of the Eighth National Conference on Artificial
  Intelligence}, pages 457 -- 464, Boston, Mass, August 1990.
\newblock ISI Technical Report \# ISI-RR-93-316.

\noindent\hspace*{\itemindent}{\leftskip=0.1in\rightskip=0.1in\hrulefill}

\bibitem{Mittal-Paris-KBCS}
Vibhu~O. Mittal and C{\'e}cile~L. Paris.
\newblock {Analogical Explanations}.
\newblock In {\em {Proceedings of the Third Conference on Knowledge Based
  Computer Systems -- {KBCS}-90}}, pages 17--26, New Dehli, India, December 13
  -- 15 1990. Center for Development of Advanced Computers.
\newblock Also published as a chapter in {\it Frontiers in Knowledge-Based
  Computing\/}, edited by V. P. Bhatkar and K. M. Rege, Narosa Publishing
  House, New Delhi, India.; ISI Technical Report \# ISI-RR-93-317.

\noindent\hspace*{\itemindent}{\leftskip=0.1in\rightskip=0.1in\hrulefill}

\bibitem{Paris-AAAIsymp90}
C\'{e}cile~L. Paris.
\newblock Tailoring as a prerequesite for effective human-computer
  communication.
\newblock In {\em Proceedings of the 1990 AAAI Symposium on Knowledge-Based
  Human Computer Communication}, Stanford, California, March 1990. American
  Association for Artificial Intelligence.
\newblock Full paper in the Proceedings of ILN-91, Nantes, France.

{\leftskip=0.1in\rightskip=0.1in\begin{small}\par\noindent{\bf Abstract:\
  }Natural Language (even if limited) is a powerful medium for interfacing with
  users or for providing documentation about a system, and sophisticated
  computer systems using natural language to interact with people are now
  becoming widespread. When humans use language, they show an essential,
  inbuilt responsiveness to their hearers/readers. They typically `tailor'
  their language to the situation and the audience. This tailoring occurs at
  all levels of linguistic expression: from the content and organization of the
  text as a whole to lexical and syntactic constructions of individual
  sentences (i.e., the phrasing of the text). Indeed, actual texts show that
  the same question can be answered differently depending on the situation and
  the expected audience, giving rise to texts with different content,
  structures and syntactic forms. For example, giving instructions will give
  rise to a different text than providing a summary of a procedure. A text for
  an expert will be different than one for a layman. When language is generated
  by machine, it is similarly necessary to ensure that that language is
  appropriate for its intended audience. It is thus necessary for computational
  systems to be capable of appropriately tailoring the texts they produce. This
  is especially important as computer systems often need to communicate with an
  increasingly varied user community, across an ever more extensive range of
  situations. In our current work, we are building an expert system explanation
  facility that must produce explanations of the expert system's behavior to a
  variety of different users -- users who differ in goals, expectations, and
  expertise concerning both the expert system and its domain. Although we do
  not want to have a system that {\em relies} on a detailed, complete, and
  correct user model, we argue that a model of the user and the situation can
  greatly improve the answers provided by a generation system, and that, in
  fact, to communicate effectively, systems need to have such a model and need
  to be able to tailor their ouput. In this talk, we will show how we can
  incorporate some of our previous work on tailoring to a user the structure
  and content of a text into our explanation system. We will also present some
  new work we are currently doing on tailoring the {\em phrasing} of a text
  according to users and situations.\end{small}\par}
\noindent\hspace*{\itemindent}{\leftskip=0.1in\rightskip=0.1in\hrulefill}

\bibitem{Chandra-Swartout}
B.~Chandrasekaran and William Swartout.
\newblock {Explanations in Knowledge Systems: The Role of Explicit
  Representation of Design Knowledge}.
\newblock {\em IEEE Expert}, 6(3):47--50, June 1991.
\newblock ISI Technical Report \# ISI-RR-93-303.

\noindent\hspace*{\itemindent}{\leftskip=0.1in\rightskip=0.1in\hrulefill}

\bibitem{EES-IEEEExpert}
William~R. Swartout, C\'{e}cile~L. Paris, and Johanna~D. Moore.
\newblock {Design for Explainable Expert Systems.}
\newblock {\em IEEE Expert}, 6(3):58--64, June 1991.
\newblock ISI Technical Report \# ISI-RR-93-304.

\noindent\hspace*{\itemindent}{\leftskip=0.1in\rightskip=0.1in\hrulefill}


\bibitem{Paris-Catalina}
C\'{e}cile~L. Paris.
\newblock {Generation and Explanation: Building an Explanation Facility for the
  Explainable Expert Systems Framework}.
\newblock In C\'{e}cile~L. Paris, William~R. Swartout, and William~C. Mann,
  editors, {\em Natural Language Generation in Artificial Intelligence and
  Computational Linguistics}, pages 49--81. Kluwer Academic Publishers, Boston,
  1991.
\newblock ISI Technical Report \# ISI-RR-93-318.

{\leftskip=0.1in\rightskip=0.1in\begin{small}\par\noindent{\bf Abstract:\
  }Generating explanations for expert systems has not been seen as a
  sophisticated generation problem in the past, and researchers working on
  expert system explanations (mainly researchers working on expert systems
  themselves) have been largely separated from the natural language generation
  community. In this paper, we argue that explanation for expert systems can
  benefit from the more sophisticated generation techniques being developed in
  computational linguistics and that explanation for expert systems actually
  provides a rich domain in which to study natural language generation. We
  describe our efforts to build a generation facility for the Explainable
  Expert Systems (EES) framework, presenting the requirements for this
  generation task and the issues addressed. We initially tried to use known
  natural language generation techniques but were led to design a new language
  for planning text, as these techniques did not fit our needs. This paper thus
  presents an overview of the generation facility being built for EES,
  including an `historical' perspective that explains the decisions we made.
  Finally, we briefly present directions for future research.\end{small}\par}
\noindent\hspace*{\itemindent}{\leftskip=0.1in\rightskip=0.1in\hrulefill}

\bibitem{BatemanParis91-helsinki}
John~A. Bateman and C\'ecile~L. Paris.
\newblock {Constraining the development of lexicogrammatical resources during
  text generation: towards a computational instantiation of register theory}.
\newblock In Eija Ventola, editor, {\em Recent Systemic and Other Views on
  Language}, pages 81--106. Mouton, Amsterdam, 1991.
\newblock ISI Technical Report \# ISI-RR-93-319; Authors in alphabetical order.

\noindent\hspace*{\itemindent}{\leftskip=0.1in\rightskip=0.1in\hrulefill}

\bibitem{MooreParisCIJ91}
Johanna~D. Moore and C\'{e}cile~L. Paris.
\newblock {Requirements for an Expert System Explanation Facility}.
\newblock {\em Computational Intelligence}, 7(4), 1991.
\newblock ISI Technical Report \# ISI-RR-93-320.

{\leftskip=0.1in\rightskip=0.1in\begin{small}\par\noindent{\bf Abstract:\ }For
  the past several years, we have worked on building an explanation component
  for an expert system building framework (or `shell'), the Explainable Expert
  System ({\sc ees}) Framework. From this experience we have identified a set
  of characteristics that we believe to be essential for an explanation
  component of an expert system and have built an explanation facility that
  strives to embody these characteristics. In this talk, we briefly describe
  these characteristics and identify the important features of our architecture
  that support the desired capabilities. \\ \\ Authors in alphabetical
  order.\end{small}\par}
\noindent\hspace*{\itemindent}{\leftskip=0.1in\rightskip=0.1in\hrulefill}

\bibitem{Moore-Swartout-wkshp88}
Johanna~D. Moore and William~R. Swartout.
\newblock A {R}eactive {A}pproach to {E}xplanation: {T}aking the {U}ser's
  {F}eedback into {A}ccount.
\newblock In C.~Paris, W.~Swartout, and W.~Mann, editors, {\em Natural Language
  Generation in Artificial Intelligence and Computational Linguistics}. Kluwer
  Academic Publishers, Boston/Dordrecht/London, 1991.
\newblock ISI Technical Report \# ISI-RR-93-321.

{\leftskip=0.1in\rightskip=0.1in\begin{small}\par\noindent{\bf Abstract:\
  }Explanation is an interactive process, requiring a dialogue between
  advice-giver and advice-seeker. Yet current expert systems cannot participate
  in a dialogue with users. In particular these systems cannot clarify
  misunderstood explanations, elaborate on previous explanations, or respond to
  follow-up questions in the context of the on-going dialogue. In this paper,
  we describe a reactive approach to explanation -- one that can participate in
  an on-going dialogue and employs feedback from the user to guide subsequent
  explanations. Our system plans explanations from a rich set of explanation
  strategies, recording the system's discourse goals, the plans used to achieve
  them, and any assumptions made while planning a response. This record
  provides the dialogue context the system needs to respond appropriately to
  the user's feedback. We illustrate our approach with examples of
  disambiguating a follow-up question and producing a clarifying elaboration in
  response to a misunderstood explanation.\end{small}\par}
\noindent\hspace*{\itemindent}{\leftskip=0.1in\rightskip=0.1in\hrulefill}

\bibitem{boy-paris}
Guy Boy and C\'ecile~L. Paris.
\newblock {An Intelligent Document Browsing System that Incorporates Indexing
  in Context}.
\newblock Technical Report, 1991.

{\leftskip=0.1in\rightskip=0.1in\begin{small}\par\noindent{\bf Abstract:\ }To
  generate intelligent indexing that allows context-sensitive information
  retrieval, a system must be able to acquire knowledge directly from users
  interacting with it. In this paper, we present the architecture we have
  developed for CID (Computer Integrated Documentation), a system that enable
  integration of various technical documents in an hypermedia framework as well
  as context-sensitive retrieval. CID includes a knowledge-based indexing
  mechanism that allows case-based knowledge acquisition by experimentation. It
  utilizes on-line user requirements and suggestions to either reinforce
  current actions in case of success or to generate new knowledge in case of
  failure. This allows CID's intelligent interface system to provide helpful
  responses, even when no robust user model is available. Our system in fact
  learns how to exploit a user model based on experience. We describe CID's
  current capabilities and provide an overview of our plans for extending the
  system.\end{small}\par}
\noindent\hspace*{\itemindent}{\leftskip=0.1in\rightskip=0.1in\hrulefill}

\bibitem{ParisMaier91}
Cecile~L. Paris and Elisabeth~A. Maier.
\newblock {Knowledge Resources or Decisions?}
\newblock In {\em IJCAI-91 Workshop on Decision Making throughout the
  Generation Process}, pages 11--17, Sydney, Australia, 1991.
\newblock ISI Technical Report \# ISI-RR-93-322.

{\leftskip=0.1in\rightskip=0.1in\begin{small}\par\noindent{\bf Abstract:\ }In
  this paper we argue that the problem of {\em decisions\/} can only be
  discussed when the {\em resources\/} which contribute to the process of text
  generation are identified. We claim that declarative and procedural knowledge
  - while resources correspond to the former and decisions to the latter - have
  to be clearly separated. After evaluating various text planning systems from
  this angle we outline what the consequences for the design of a text
  generation system are.\end{small}\par}
\noindent\hspace*{\itemindent}{\leftskip=0.1in\rightskip=0.1in\hrulefill}

\bibitem{CahourParis91}
B\'eatrice Cahour and C\'{e}cile~L. Paris.
\newblock {Role and Use of User Models}.
\newblock In the {\em Proceedings of the IJCAI-91 Workshop on Agent Modelling
  for Intelligent Interaction\/}; ISI Technical Report \# ISI-RR-93-323, August
  1991.

{\leftskip=0.1in\rightskip=0.1in\begin{small}\par\noindent{\bf Abstract:\ }Our
  objective in this paper is to stress the importance of studying user modeling
  within its context of use and to start a characterization of the links
  between the types of user models available and their intended role in a
  system. We discuss various tasks which engage a user in a dialog, examine the
  role a user model plays in these tasks and the type of user models these
  roles imply. The characterization of the links between the roles a user model
  can play in an interaction and the type of models which will support this
  role is a crude one at this point, but we believe it is an important
  beginning, as this identification process will become very important if user
  models are to be used in practical systems.\\ \\ Authors in alphabetical
  order.\end{small}\par}
\noindent\hspace*{\itemindent}{\leftskip=0.1in\rightskip=0.1in\hrulefill}

\bibitem{Moore-Paris-aaai91}
Johanna~D. Moore and C{\'e}cile~L. Paris.
\newblock {The EES Explanation Facility: its Tasks and its Architecture}.
\newblock In {\em {Proceedings of the AAAI'91 Workshop on Comparative Analysis
  of Explanation Planning Architectures}}, Anaheim, Ca, July 1991. American
  Association for Artificial Intelligence.
\newblock Authors in alphabetical order. Extended Abstract.

\noindent\hspace*{\itemindent}{\leftskip=0.1in\rightskip=0.1in\hrulefill}

\bibitem{Cecile-91}
C\'{e}cile~L. Paris.
\newblock {The role of the user's domain knowledge in generation}.
\newblock {\em Computational Intelligence}, 7 (2):71--93, May 1991.
\newblock This is an extended version of a paper which appears in the
  Proceedings of the International Computer Science Conference '88, sponsored
  by IEEE; ISI Technical Report \# ISI-RR-93-324.

{\leftskip=0.1in\rightskip=0.1in\begin{small}\par\noindent{\bf Abstract:\ }A
  question answering program that provides access to a large amount of data
  will be most useful if it can tailor its answers to each individual user. In
  particular, a user's level of knowledge about the domain of discourse is an
  important factor in this tailoring if the answer provided is to be both
  informative and understandable to the user. In this research, we address the
  issue of how the user's domain knowledge, or the level of expertise, might
  affect an answer. We present our generation system, TAILOR, which uses
  information about a user's level of expertise to combine discourse strategies
  in a single text, choosing the most appropriate at each point in the
  generation process, in order to generate texts for users anywhere along the
  knowledge spectrum from naive to expert, without a predefined set of
  stereotypes\end{small}\par}
\noindent\hspace*{\itemindent}{\leftskip=0.1in\rightskip=0.1in\hrulefill}

\bibitem{Moore-Paris-AAAIsymp91}
Johanna~D. Moore and C\'{e}cile~L. Paris.
\newblock {Discourse Structure for Explanatory Dialogues}.
\newblock In {\em Proceedings of the AAAI-91 Fall Symposium on Discourse
  Structure in Natural Language Understanding and Generation}, Asilomar,
  California, November 1991. American Association for Artificial Intelligence.
\newblock Authors in alphabetical order. Extended Abstract.

\noindent\hspace*{\itemindent}{\leftskip=0.1in\rightskip=0.1in\hrulefill}

\bibitem{Paris91-survey-misc}
C\'ecile~L. Paris.
\newblock Text generation: a survey paper, 1991.
\newblock ISI Technical Report \# ISI-RS-93-314.

\noindent\hspace*{\itemindent}{\leftskip=0.1in\rightskip=0.1in\hrulefill}

\bibitem{Bateman-Paris-methodology}
C\'{e}cile~L. Paris and John~A. Bateman.
\newblock {A Methodology for Investigating Registers}, 1992.

{\leftskip=0.1in\rightskip=0.1in\begin{small}\par\noindent{\bf Abstract:\ }When
  humans use language, they show an essential responsiveness to their hearers.
  When language is automatically generated, it is similarly necessary to ensure
  that that language is appropriate for its intended audience. In previous
  work, we suggested that register, as a body of linguistic work that claims to
  deal precisely with the inter-relationship between linguistic variation and
  types of audience and situations, could provide significant input to problems
  of text tailoring, user modeling, and stylistic control of a grammar's output
  during text generation. In this paper, we outline how some recent advances in
  promise to support a more rigorous and non-{\em ad hoc} body of linguistic
  work on register. We show how this work can in turn feed back into generation
  research to start providing the theoretical benefits and practical
  improvements in functionality that we have previsouly argued register theory
  offers. In particular, we outline how a register construction workbench can
  be designed so as to streamline register analysis while simultaneously
  constructing resources that may be used by text generation
  systems.\end{small}\par}
\noindent\hspace*{\itemindent}{\leftskip=0.1in\rightskip=0.1in\hrulefill}

\bibitem{Moore-Paris-UM}
Johanna~D. Moore and C\'{e}cile~L. Paris.
\newblock {Exploiting User Feedback to Compensate for the Unreliability of User
  Models}.
\newblock {\em User Model and User Adapted Interaction Journal}, 2(4), 1992.
\newblock ISI Technical Report \# ISI-RR-93-326.

{\leftskip=0.1in\rightskip=0.1in\begin{small}\par\noindent{\bf Abstract:\ }To
  participate in a dialogue a system must be capable of reasoning about its own
  previous utterances. Follow-up questions must be interpreted in the context
  of the ongoing conversation, and the system's previous contributions form
  part of this context. Furthermore, knowing what it has said previously is
  essential if a system is to be able to clarify misunderstood explanations or
  to elaborate on prior explanations. Previous approaches to generating
  multisentential texts have relied solely on rhetorical structuring
  techniques. In this paper, we argue that to handle explanation dialogues
  successfully, a discourse model must include information about the intended
  effect of individual parts of the text on the hearer as well as how the parts
  relate to one another rhetorically.\\ \\ Authors in alphabetical
  order.\end{small}\par}
\noindent\hspace*{\itemindent}{\leftskip=0.1in\rightskip=0.1in\hrulefill}

\bibitem{Mittal-ACL92}
Vibhu~O. Mittal.
\newblock {Elaboration in Object Descriptions through Examples}.
\newblock In {\em Proceedings of the 30th Annual Meeting of the Association for
  Computational Linguistics (ACL-92)}, pages 315--317, Newark, Delaware, 1992.
\newblock (Student Session).

{\leftskip=0.1in\rightskip=0.1in\begin{small}\par\noindent{\bf Abstract:\
  }Examples are often used along with textual descriptions to help convey
  particular ideas - especially in instructional or explanatory contexts. These
  accompanying examples reflect information in the surrounding text, and in
  turn, also influence the text. Sometimes, examples replace possible (textual)
  elaborations in the description. It is thus clear that if object descriptions
  are to be generated, the system must incorporate strategies to handle
  examples. In this work, we shall investigate some of these issues in the
  generation of object descriptions.\end{small}\par}
\noindent\hspace*{\itemindent}{\leftskip=0.1in\rightskip=0.1in\hrulefill}

\bibitem{Mittal-Paris-CAIC92}
Vibhu~O. Mittal and C{\'e}cile~L. Paris.
\newblock {Generating Object Descriptions which integrate both Text and
  Examples}.
\newblock In {\em Proceedings of the Ninth Canadian Artificial Intelligence
  Conference (AI/GI/VI 92)}, pages 1--8, Vancouver, Canada, 1992. Canadian
  Society for the Computational Studies of Intelligence (CSCSI), Morgan
  Kaufmann Publishers.
\newblock ISI Technical Report \# ISI-RR-93-327.

{\leftskip=0.1in\rightskip=0.1in\begin{small}\par\noindent{\bf Abstract:\
  }Descriptions of complex concepts often use examples for illustrating various
  points. This paper discusses the issues that arise in generating complex
  descriptions in tutorial contexts. Although some tutorial systems have used
  examples in explanations, they have rarely been considered as an integral
  part of the complete explanation -- they have usually been merely supportive
  devices -- and inserted in the explanations without any representation in the
  system of how the examples relate to and complement the textual explanations
  that accompany the examples. This can lead to presentations that are at best,
  weakly coherent, and at worst, confusing and mis-leading for the learner. In
  this paper, we consider the generation of examples as an integral part of the
  overall process of generation, resulting in examples and text that are
  smoothly integrated and complement each other. We address the requirements of
  a system capable of this, and present a framework in which it is possible to
  generate examples as an integral part of a description. We then show how
  techniques developed in Natural Language Generation can be used to build such
  a framework.\end{small}\par}
\noindent\hspace*{\itemindent}{\leftskip=0.1in\rightskip=0.1in\hrulefill}

\bibitem{Mittal-Paris-SpringSymp92}
Vibhu~O. Mittal and C{\'e}cile~L. Paris.
\newblock {Producing Cooperative Explanations using Examples}.
\newblock In Alex Quilici, editor, {\em Proceedings of the AAAI Spring
  Symposium on Producing Cooperative Explanations}, pages 39--45, Palo Alto,
  CA., 1992. American Association for Artificial Intelligence.
\newblock Extended Abstract.

\noindent\hspace*{\itemindent}{\leftskip=0.1in\rightskip=0.1in\hrulefill}

\bibitem{Hovy-etal92-nlgw}
Eduard Hovy, Julia Lavid, Elisabeth Maier, Vibhu Mittal, and Cecile Paris.
\newblock {Employing Knowledge Resources in a New Text Planner Architecture}.
\newblock In {\em Proceedings of the 6th International Workshop on Natural
  Language Generation}, pages 57--73, Trento, Italy, 1992. Springer-Verlag.
\newblock ISI Technical Report \# ISI-RR-93-328.

{\leftskip=0.1in\rightskip=0.1in\begin{small}\par\noindent{\bf Abstract:\ }We
  describe in this paper a new text planner that has been designed to address
  several problems we had encountered in previous systems. Motivating factors
  include a clearer and more explicit separation of the declarative and
  procedural knowledge used in a text generation system as well as the
  identification of the distinct types of knowledge necessary to generate
  coherent discourse, such as communicative goals, text types, schemas,
  discourse structure relations, and theme development patterns. This knowledge
  is encoded as separate resources and integrated under a flexible planning
  process that draws from appropriate resources whatever knowledge is needed to
  construct a text. We describe the resources and the planning process and
  illustrate the ideas with an example. \\ \\ Authors in alphabetical
  order.\end{small}\par}
\noindent\hspace*{\itemindent}{\leftskip=0.1in\rightskip=0.1in\hrulefill}

\bibitem{Mittal-Paris-CogSci92}
Vibhu~O. Mittal and C{\'e}cile~L. Paris.
\newblock {Finding and Using Analogies in Generating Natural Language Object
  Descriptions}.
\newblock In {\em Proceedings of the Fourteenth Annual Conference of The
  Cognitive Science Society}, pages 996--1002, Indianapolis, IN., August 1992.
  Lawrence Erlbaum, Inc.
\newblock ISI Technical Report \# ISI-RR-93-329.

{\leftskip=0.1in\rightskip=0.1in\begin{small}\par\noindent{\bf Abstract:\ }The
  ability to generate descriptive explanations of domain concepts defined in a
  knowledge base is an important requirement for any system with explanatory
  capabilities. The ability to use analogies to highlight selected features in
  the description of the concept greatly enhances the possibility of the system
  being able to convey its point to the user. In this paper, we describe a
  system designed within the EES framework that embodies this capability.
  Finding analogies is not simple, but we shall show how the discourse
  structure provides the system with additional knowledge that aids finding an
  acceptable analogy to express. Analogies are a powerful and compact means of
  communicating ideas and descriptions. Using analogies in language generation
  is different from using analogies in problem solving. This paper will outline
  some of these differences and demonstrates one attempt at incorporating them
  within an expert system that generates explanations for its users in natural
  language.\end{small}\par}
\noindent\hspace*{\itemindent}{\leftskip=0.1in\rightskip=0.1in\hrulefill}

\bibitem{Paris-French92}
C\'{e}cile~L. Paris.
\newblock G\'en\'eration et explications: Le module d'explications dans
  l'architecture de l'explainable expert system.
\newblock {\em Langages}, 106:63--74, June 1992.

\noindent\hspace*{\itemindent}{\leftskip=0.1in\rightskip=0.1in\hrulefill}

\bibitem{Paris-thesis-book}
C\'{e}cile~L. Paris.
\newblock {\em {The Use of Explicit User Models in Text Generation}}.
\newblock Frances Pinter, London, England, 1993.

{\leftskip=0.1in\rightskip=0.1in\begin{small}\par\noindent{\bf Abstract:\ }A
  question answering program that provides access to a large amount of data
  will be most useful if it can tailor its answers to each individual user. In
  particular, a user's level of knowledge about the domain of discourse is an
  important factor in this tailoring if the answer provided is to be both
  informative and understandable to the user. In this research, we address the
  issue of how the user's domain knowledge, or the level of expertise, might
  affect an answer. By studying texts we found that the user's level of domain
  knowledge affected the kind of information provided and not just the amount
  of information, as was previously assumed. Depending on the user's assumed
  domain knowledge, a description of a complex physical objects can be either
  parts-oriented or process-oriented. Thus the user's level of expertise in a
  domain can guide a system in choosing the appropriate facts from the
  knowledge base to include in an answer. We propose two distinct descriptive
  strategies that can be used to generate texts aimed at naive and expert
  users. Users are not necessarily truly expert or fully naive however, but can
  be anywhere along a knowledge spectrum whose extremes are naive and expert.
  In this work, we show how our generation system, TAILOR, can use information
  about a user's level of expertise to combine several discourse strategies in
  a single text, choosing the most appropriate at each point in the generation
  process, in order to generate texts for users anywhere along the knowledge
  spectrum. TAILOR's ability to combine discourse strategies based on a user
  model allows for the generation of a wider variety of texts and the most
  appropriate one for the user.\end{small}\par}
\noindent\hspace*{\itemindent}{\leftskip=0.1in\rightskip=0.1in\hrulefill}

\bibitem{Moore-Paris-CL}
Johanna~D. Moore and C\'{e}cile~L. Paris.
\newblock {Planning Text for Advisory Dialogues: Capturing Intentional, and
  Rhetorical Information}.
\newblock {\em Computational Linguistics}, 19 (4):651--694, December 1993.
\newblock ISI Technical Report \# RS 330 and Technical Report from the
  University of Pittsburgh, Department of Computer Science (Number 92--22).

{\leftskip=0.1in\rightskip=0.1in\begin{small}\par\noindent{\bf Abstract:\ }To
  participate in a dialogue a system must be capable of reasoning about its own
  previous utterances. Follow-up questions must be interpreted in the context
  of the ongoing conversation, and the system's previous contributions form
  part of this context. Furthermore, if a system is to be able to clarify
  misunderstood explanations or to elaborate on prior explanations, it must
  understand what is has conveyed in prior explanations. Previous approaches to
  generating multisentential texts have relied solely on rhetorical structuring
  techniques. In this paper, we argue that, to handle explanation dialogues
  successfully, a discourse model must include information about the intended
  effect of individual parts of the text on the hearer, as well as how the
  parts relate to one another rhetorically. We present a text planner that
  records this information, and show how the resulting structure is used to
  respond appropriately to a follow-up question.\\ \\ Authors in alphabetical
  order.\end{small}\par}
\noindent\hspace*{\itemindent}{\leftskip=0.1in\rightskip=0.1in\hrulefill}

\bibitem{Mittal-Paris-AAAI93}
Vibhu~O. Mittal and C{\'e}cile~L. Paris.
\newblock {Generating Natural Language Descriptions with Examples: Differences
  between introductory and advanced texts}.
\newblock In {\em Proceedings of the {\it Eleventh National Conference on
  Artificial Intelligence -- AAAI 93\/}}. American Association for Artificial
  Intelligence, 1993.
\newblock ISI Technical Report \# ISI-RR-93-331.

{\leftskip=0.1in\rightskip=0.1in\begin{small}\par\noindent{\bf Abstract:\
  }Examples form an integral and very important part of many descriptions,
  especially in contexts such as tutoring and documentation generation. The
  ability to tailor a description for a particular situation is particularly
  important when different situations can result in widely varying
  descriptions. This paper considers the generation of descriptions with
  examples for two different situations: introductory texts and advanced,
  reference manual style texts. Previous studies have focused on any the
  examples or the language component of the explanation in isolation. However,
  there is a strong interaction between the examples and the accompanying
  description and it is therefore important to study how both these components
  are affected by changes in the situation. In this paper, we characterize
  examples in the context of their description along three orthogonal axes: the
  information content, the knowledge type of the example and the text-type in
  which the explanation is being generated. While variations along either of
  the three axes can result in different descriptions, this paper addresses
  variation along the text-type axis. We illustrate our discussion with a
  description of a {\tt list} from our domain of {\sc lisp} documentation, and
  present a trace of the system as it generates these
  descriptions.\end{small}\par}
\noindent\hspace*{\itemindent}{\leftskip=0.1in\rightskip=0.1in\hrulefill}

\bibitem{Mittal-Paris-Education93}
Vibhu~O. Mittal and C{\'e}cile~L. Paris.
\newblock {Categorizing Example Types in Instructional Texts: The Need to
  consider Context}.
\newblock In {\em Proceedings of {\it AI-ED 93: World Conference on Artificial
  Intelligence in Education\/}}, Edinburgh, Scotland, 1993. AACE.
\newblock ISI Technical Report \# ISI-RR-93-332.

{\leftskip=0.1in\rightskip=0.1in\begin{small}\par\noindent{\bf Abstract:\
  }Different situations often require the presentation of different types of
  examples with specific presentation requirements about the number of
  examples, the sequence of presentation, the associated prompts, etc. A
  specification of the different presentation requirements is particularly
  important in designing an effective ITS. A categorization of example types
  and their associated presentation requirements is necessary. In this paper,
  we argue that examples must be characterized based on the context in which
  they appear, and present one such characterization, and describe how these
  can be effectively used by an ITS to generate tutorial descriptions that
  incorporate examples.\end{small}\par}
\noindent\hspace*{\itemindent}{\leftskip=0.1in\rightskip=0.1in\hrulefill}

\bibitem{Mittal-Paris-IJCAI93}
Vibhu~O. Mittal and C{\'e}cile~L. Paris.
\newblock {Automatic Documentation Generation: The Interaction between Text and
  Examples}.
\newblock In {\em Proceedings of {\it Thirteenth International Joint Conference
  on Artificial Intelligence\/}}, Chambery, France, 1993.
\newblock ISI Technical Report \# ISI-RR-93-333.

{\leftskip=0.1in\rightskip=0.1in\begin{small}\par\noindent{\bf Abstract:\ }Good
  documentation is critical for user acceptance of any system, and empirical
  studies have shown that examples can greatly increase effectiveness of system
  documentation. However, studies also show that badly integrated text and
  examples can be actually detrimental compared to using either text or
  examples alone. It is thus clear that in order to provide useful
  documentation automatically, a system must be capable of providing
  well-integrated examples to illustrate its points. Previous work on example
  generation has concentrated on the issue of retrieving or constructing
  examples. In this paper, we look at the {\em integration\/} of text and
  examples. We identify how text and examples co-constrain each other and show
  that a system must consider example generation as an integral part of the
  generation process. Finally, we present such a system, together with an
  example.\end{small}\par}
\noindent\hspace*{\itemindent}{\leftskip=0.1in\rightskip=0.1in\hrulefill}

\bibitem{mittal-paris-context}
Vibhu~O. Mittal and C{\'e}cile~L. Paris.
\newblock {Context: Its elements from the viewpoint of communication}.
\newblock In the {\it Proceedings of the IJCAI-93 Workshop on Context}
  (Chambery, France), August 29 1993.

{\leftskip=0.1in\rightskip=0.1in\begin{small}\par\noindent{\bf Abstract:\
  }Context is an important aspect that must be considered in any communicative
  system. Various people have used the notion of context in different ways. In
  this paper, we attempt to bring together these different notions of context
  as elements of a global picture. We characterize these elements as belonging
  to one of five categories: $(a)$ the problem solving situation, $(b)$ the
  participants involved, $(c)$ the mode of interaction, $(d)$ the discourse,
  and $(e)$ the external world. We examine each of these categories from the
  view-point of communication, further refining them, and present examples to
  illustrate the points being made. \\ \\ Authors in alphabetical
  order.\end{small}\par}
\noindent\hspace*{\itemindent}{\leftskip=0.1in\rightskip=0.1in\hrulefill}

\bibitem{mittal-paris-critiquing}
Vibhu~O. Mittal and C{\'e}cile~L. Paris.
\newblock {Text Generation: Explanation vs Criticism in Expert Systems}.
\newblock In the {\it Proceedings of the AAAI-93 Workshop on Expert Critiquing
  Systems} (Washington, DC), July 1993.
\newblock Authors in Alphabetical Order. Extended Abstract.

\noindent\hspace*{\itemindent}{\leftskip=0.1in\rightskip=0.1in\hrulefill}

\bibitem{Mittal-Paris-HCI93}
Vibhu~O. Mittal and C{\'e}cile~L. Paris.
\newblock {Building Intelligent Help Facilities: Generating Natural language
  Descriptions with examples}.
\newblock In Gavriel Salvendy and Michael~J. Smith, editors, {\em
  Human-Computer Interaction: Software and Hardware Interfaces}, pages
  379--384, Orlando, FL, August 1993. Elsevier.
\newblock (Proceedings of the 5th International Conference on Human-Computer
  Interaction. Also available as USC/ISI Research Report \#ISI/RR-93-334).

{\leftskip=0.1in\rightskip=0.1in\begin{small}\par\noindent{\bf Abstract:\
  }On-line help facilities are essential in any complex system, especially for
  introductory or naive users. Previous studies have highlighted the need for
  appropriate examples along with the description. This paper describes a
  help/documentation facility built within an explanation framework that plans
  the presentation of text and examples using techniques in natural language
  generation. The paper shows how text and examples can influence each other
  and enumerates some of the other issues that arise in planning such
  presentations.\end{small}\par}
\noindent\hspace*{\itemindent}{\leftskip=0.1in\rightskip=0.1in\hrulefill}

\bibitem{mittal-paris-cogsci93}
Vibhu~O. Mittal and C{\'e}cile~L. Paris.
\newblock {A Categorization of Example Types and their applications for the
  Generation of Tutorial Descriptions}.
\newblock In {\em Proceedings of Fifteenth Annual Conference of the Cognitive
  Science Society (CogSci-93)}, Boulder, Colorado, June 18--21 1993. Cognitive
  Science Society.
\newblock ISI Technical Report \# ISI-RR-93-335.

{\leftskip=0.1in\rightskip=0.1in\begin{small}\par\noindent{\bf Abstract:\
  }Different situations may require the presentation of different types of
  examples. For instance, some situations require the presentation of positive
  examples only, while others require both positive and negative examples.
  Furthermore, different examples often have specific presentation
  requirements: they need to appear in an appropriate sequence, be introduced
  properly and often require associated prompts. It is important to be able to
  identify what is needed in which case, and what needs to be done in
  presenting the example. A categorization of examples, along with their
  associated presentation requirements would help tremendously. This issue is
  particularly salient in the design of a computational framework for the
  generation of tutorial descriptions which include examples. Previous work on
  characterizing examples has approached the issue from the direction of {\it
  when\/} different types of examples should be provided, rather than {\it
  what\/} characterizes the different types. In this paper, we extend previous
  work on example characterization in two ways: (i) we show that the scope of
  the characterization must be extended to include not just the example, but
  also the surrounding context, and (ii) we characterize examples in terms of
  three orthogonal dimensions: the {\it information content\/}, the {\it
  intended audience\/}, and the {\it knowledge type\/}. We present descriptions
  from text-books on {\sc lisp} to illustrate our points, and describe how such
  categorizations can be effectively used by a computational system to generate
  descriptions that incorporate examples.\end{small}\par}
\noindent\hspace*{\itemindent}{\leftskip=0.1in\rightskip=0.1in\hrulefill}

\bibitem{Mittal-Paris-PacLing93}
Vibhu~O. Mittal and C{\'e}cile~L. Paris.
\newblock {The Placement of Examples in Descriptions: Before, Within or After
  the Text}.
\newblock In {\em Proceedings of {\it First Pacific Association for
  Computational Linguistics Conference\/}}, pages 279--287, Vancouver, Canada,
  May 1993.
\newblock ISI Technical Report \# ISI-RR-93-336.

{\leftskip=0.1in\rightskip=0.1in\begin{small}\par\noindent{\bf Abstract:\
  }Examples are often integral to explanations, especially in contexts such as
  instruction and the generation of automatic documentation. There are many
  issues that must be addressed by generation systems attempting to produce
  coherent, integrated descriptions that incorporate examples along with the
  descriptive explanation. One such issue is the {\it positioning\/} of the
  examples with respect to the accompanying descriptive explanation. There are
  three possibilities: $(i)$ the example(s) occur {\it before\/} the
  description, $(ii)$ the example(s) occur {\it within\/} the description, and
  $(iii)$ the example(s) occur {\it after\/} the description. There are
  implications of these different placement strategies on the generation of not
  only the examples, but on the textual explanation itself. It is thus
  important for a generation system to be able to present the example(s)
  correctly with respect to the accompanying description. In this paper, we
  present a simple, yet effective, algorithm based on two factors: the {\it
  text-type\/} being generated, and the {\it communicative goal\/} being
  achieved to explain the different positions that examples are observed
  in.\end{small}\par}
\noindent\hspace*{\itemindent}{\leftskip=0.1in\rightskip=0.1in\hrulefill}

\bibitem{Mittal93-Thesis}
Vibhu~O. Mittal.
\newblock {\em {Generating Natural Language Descriptions with Integrated Text
  and Examples}}.
\newblock PhD thesis, University of Southern California, Los Angeles,
  California, 1993.

{\leftskip=0.1in\rightskip=0.1in\begin{small}\par\noindent{\bf
Abstract:\ }Good documentation is critical for user acceptance of any
system. Advances in areas such as knowledge-based systems, text
generation and multi-media have now made it possible to investigate the
automatic generation of documentation from the underlying knowledge
bases. Empirical studies have shown that examples can greatly increase
the effectiveness of system documentation. However, studies also show
that badly integrated text and examples can be actually detrimental
compared to using either text or examples alone. It is thus clear that
in order to provide useful documentation automatically, a system must be
capable of providing well-integrated examples to illustrate its points.
This thesis builds upon previous work in natural language generation,
example generation, cognitive science and educational psychology to
identify relevant issues in the generation of coherent descriptions that
integrate text and examples. We identify how text and examples
co-constrain each other and show that a system must consider example
generation as an integral part of the generation process. We describe an
implementation, and present an initial evaluation of the system's
effectiveness.\end{small}\par}
  
\noindent\hspace*{\itemindent}{\leftskip=0.1in\rightskip=0.1in\hrulefill}


\bibitem{Mittal-Paris-esa-journal-94}
Vibhu~O. Mittal and C{\'e}cile~L. Paris.
\newblock {Generating Explanations in Context: The System's Perspective}.
\newblock {\em Expert Systems with Applications}, 8 (4):491--503, 1995.
Special Issue on Explanation, Diane Berry, special editor.

{\leftskip=0.1in\rightskip=0.1in\begin{small}\par\noindent{\bf Abstract:\ 
  }Explanations for expert systems are best provided in
context, and, recently, many systems have used some notion of context in
different ways in their explanation module. For example, some
explanation systems take into account a user model. Others generate an
explanation depending on the preceding and current discourse. In this
paper, we bring together these different notions of context as elements
of a global picture that might be taken into account by an explanation
module, depending on the needs of the application and the user. We
characterize each of these elements, describe the constraints they place
on communication and present examples to illustrate the points being
made. We discuss the implications of these different aspects of context
on the design of explanation facilities.  Finally, we describe and
illustrate with examples, an implemented intention-based planning
framework for explanation that can take into account the different
aspects of context discussed above.
  \end{small}\par}
\noindent\hspace*{\itemindent}{\leftskip=0.1in\rightskip=0.1in\hrulefill}



\bibitem{Swartout-Moore-second-generation-book}
William~R. Swartout and Johanna~D. Moore
\newblock {Explanation in Second Generation Expert Systems}.
\newblock In {\em Second Generation Expert Systems}, J. David, J.
Krivine, and R. Simmons (eds), Springer-Verlag, Berlin, 1993.

{\leftskip=0.1in\rightskip=0.1in\begin{small}\par\noindent{\bf Abstract:\ 
  }What is needed for good explanation?  This paper begins by considering some
desiderata for expert system explanation.  These desiderata concern not
only the form and content of the explanations, but also the impact of
explanation generation on the expert system itself--how it is built and how
it performs.  In this paper, we use these desiderata as a yardstick for
measuring progress in the field.  The paper describes two major
developments that have differentiated explanation in second generation
systems from explanation in first generation systems: 1) new architectures
have been developed that capture more of the knowledge that is needed for
explanation, and 2) more powerful explanation generators have been
developed in which explanation generation is viewed as a problem-solving
activity in its own right.  These developments have led to significant
improvements in explanation facilities: the explanations they offer are
richer and more coherent, they are better adapted to the user's needs and
knowledge, and the explanation facilities can offer clarifying explanations
to correct misunderstandings.
  \end{small}\par}
\noindent\hspace*{\itemindent}{\leftskip=0.1in\rightskip=0.1in\hrulefill}


\end{thebibliography}






\end{document}